\documentclass[12pt]{report}
\usepackage{amsmath}
\usepackage{amsfonts}
\usepackage{amssymb}
\usepackage{mathrsfs}
\usepackage{graphicx}
\usepackage{amsthm}
\usepackage[none]{hyphenat}
%\usepackage{cancel}

%\usepackage{times}
\usepackage{hyperref}
%\usepackage{setspace}
%\usepackage[nottoc,notlof,notlot]{tocbibind} 
%\renewcommand\bibname{References}
\usepackage[english]{babel}
\addto{\captionsenglish}{%
  \renewcommand{\bibname}{References}
}
\DeclareMathAlphabet{\mathpzc}{OT1}{pzc}{m}{it}

%\hfuzz5pt
\theoremstyle{theorem}
\newtheorem{theorem}{Theorem}
\newtheorem{corollary}[theorem]{Corollary}
\newtheorem{definition}[theorem]{Definition}
\newtheorem{example}[theorem]{Example}
\newtheorem{exercise}[theorem]{Exercise}
\newtheorem{lemma}[theorem]{Lemma}
\newtheorem{proposition}[theorem]{Proposition}
\newtheorem{remark}[theorem]{Remark}

\newcommand*\rfrac[2]{{}^{#1}\!/_{#2}}

\renewcommand{\baselinestretch}{1.8}
%%%%%%%%%%%%%%%%%%%%%%%%%%%%%%%%%%%%%%%%%%%%%%%%%%%%%%%%%%%%%%%%%%%%%%%%%%%%%%%%%%%%%%%%%%%%%%%%%DOCUMENT STARTS HERE%%%%%%%%%%%%%%%%%%%%%%%%%%%%%%%%%%%%%%%%%%%%%%%%%%%%%%%%%%%%%%%%%%%%%%%%%%%%%%%%%%%%%%%%%%%%%%%%%%%%%%%%%%%%%%
\begin{document}
\section*{The homogeneous coordinate ring of Grassmannian $(G_{r_{k,n}})$}

For $(v_1,v_2,\ldots,v_r)$ in $V^r$. Let
$$
v_j = \sum_{i = 1}^n x_{ij}e^i,~~~~ 1\leq j\leq r.
$$
The $x_{ij}$'s are coordinate functions on the affine space $V^r\equiv M(n,r)$ and the polynomial $k$- algebra $k[x_{ij}]$ is the coordinate ring of $V^r$. The morphism
$$
\hat{\pi}: V^r\longrightarrow \Lambda^rV
$$
induces a $k-$ algebra homomorphism between the coordinate rings
$$\hat{\pi}^*: k[x_\alpha]\longrightarrow k[x_{ij}]$$
defined by $\hat{\pi}^*(x_\alpha) = p_{\alpha}$.\\
where
$$
p_{\alpha} = \pm\det
\begin{bmatrix}
x_{\alpha_1}1 & x_{\alpha_1}2 & \cdots & x_{\alpha_1}r\\
x_{\alpha_2}1 & x_{\alpha_2}2 & \cdots & x_{\alpha_2}r\\
\vdots\\
x_{\alpha_r}1 & x_{\alpha_r}2 & \cdots & x_{\alpha_r}r\\ 
\end{bmatrix}
$$ 
the determinant of the $\alpha^{th}$ minor (with its sign) in the $n\times r$ matrix $[x_{ij}]$. Since $\ker\hat{\pi}^*$ is the ideal of $\mbox{im}~\hat{\pi}^*$, the cone over $G_{r_{k,n}}$, it is the homogeneous ideal of $G_{r_{k,n}}$ in $k[x_\alpha]$, thus the homogeneous coordinate ring of $G_{r_{k,n}}$ can be identified with the $k$- subalgebra of $k[x_{ij}]$ generated by $p_\alpha ~~~\alpha\in  l(r,n)$ and $G_{r_{k,n}}$ has a natural scheme structure defined by its homogeneous coordinate ring $k[p_\alpha]$, namely, $\mbox{proj}\quad k[p_\alpha]$. Indeed $G_{r_{k,n}}$ is a closed integral subscheme of $\mathbb{P}(\Lambda^rV)$.
\begin{remark}
\normalfont Grassmannian is describe as an $r$- dimensional linear subspaces of $V$ endowed with the structure of a variety as follows. Let
\begin{align*}
V^r &= \underbrace{V\oplus V\oplus V\oplus \cdots\oplus V}_{r-\mbox{copies}}\\
&\cong M(n,r), \mbox{  the set of } r\times n \mbox{  matrices}\\
\mbox{and  } V^{r,0} &= \{(v_1,v_2,\ldots,v_r)\in V^r|v_1,v_2,\ldots,v_r \mbox{  are linearly independent  }\}\\
&\cong  M(n,r)^\circ, \mbox{  the set of  } n \times r \mbox{  matrices of rank  } r.
\end{align*}
Clearly the column vectors of a matrix in $M(n,r)^\circ$ generate an element of $G_{r_{k,n}}$.
\end{remark}

\section*{Grassmannian as a Projective variety}
We realize $G_{r_{k,n}}$ as a projective variety by embedding it in the projective space $\mathbb{P}(\Lambda^rV).$\\
Let us consider the morphism
$$
\hat{\pi}^*:V^r \longrightarrow \Lambda^r
$$
defined by $\hat{\lambda}(v_1,v_2,\ldots,v_r) = v_1\Lambda v_2\Lambda \cdots\Lambda v_r$. Restricting $\hat{\pi}$ to $v^{r0}$ we have 
$$
\pi: V^{r,0}\longrightarrow \mathbb{P}(\Lambda^rV)
$$
This map induces another morphism
$$
\tilde{\pi}:  G_{r_{k,n}}\longrightarrow \mathbb{P}(\Lambda^rV)
$$
Thus we show that the im $\tilde{\pi}$ is a closed subset of $\mathbb{P}(\Lambda^rV)$. Let $\mathcal{T}_{\beta} = \{(\omega_\alpha)\in\mathbb{P}(\Lambda^rV)|\omega_\beta  = 1\}$ for $\beta\in I(k,n)$\\
Since $\{\mathcal{T}_\beta; \beta\in I(k,n)\}$ is an open covering of $\mathbb{P}(\Lambda^rV)$, then we demonstrate that for $\beta$ in $I(k,n)$, $\mathcal{T}_\beta\cap~\mbox{im}\tilde{\pi}$ is closed in $\mathcal{T}_\beta$. Suppose $\beta$ is arbitrary chosen, we see that $\mathcal{T}_\beta\cap \mbox{im}\tilde{\pi}(\nu_\beta) = \tilde{\pi}(\nu_\beta).$ This implies that $\mathcal{T}_\beta\cap\mbox{im}\tilde{\pi}$ consists of elements $(\omega_\alpha)\in\mathbb{P}(\Lambda^rV)$ and $\omega_\alpha$ is defined as the determinant of the $\alpha^{th}$ minor of an element in $\nu_\beta$ where $\nu_\beta$ consists of $k\times n$ matrices form.\\
$\binom{I_k}{A}$ with $I_k = k\times k$ identity matrix and $A = [a_{ij}]\in M(n-k,k)$. The element in the intersection above form an equivalent class which has the form 
$$
(\ldots,a_{ij},\ldots,f_\lambda(a_{ij}),\ldots)
$$
where $a_{ij}$ are arbitrary and $f_{\lambda}$ are the polynomial fuction on $a_{ij}$. Thus $\mbox{im}~\tilde{\pi}$ is a projective variety in $\mathbb{P}(\Lambda^rV)$ since the $\mathcal{T}_\beta\cap \mbox{ im }\tilde{\pi}$ is closed in the affine space $\mathcal{T}_\beta$.\\
It also follows that
$$
\tilde{\pi}: \nu_\beta\longrightarrow \tilde{\pi}(\nu_\beta)
$$
is an isomorphism of affine varieties and $\tilde{\pi}^{-1}(\tilde{\pi}(\nu_\pi)) = \nu_\beta$.\\
It implies that $\tilde{\pi}:G_{r_{k,n}} \longrightarrow \mbox{
im}\tilde{\pi}$ is an isomorphism, in particular local isomorphism. This $\tilde{\pi}$ embeds $G_{r_{k,n}}$ as a projective variety in $\mathbb{P}(\Lambda^rV)$. This map is called the pl$\ddot{\mbox{u}}$cker embedding map and the coordinates of its image are called the Pl$\ddot{\mbox{u}}$cker coordinates of $G_{r_{k,n}}$.

\section*{Decomposition of $G_{r_{k,n}}$ into cells.}
Gransmannian can be decomposed into different cells namely; Matroid cells, Schbert cells and Positroid cells.

\section*{Schubert Cell Decomposition}
\begin{definition}[The Schubert cells.]
\normalfont fix a full flag $\{0\} = v_0\subset v_1\subset\cdots\subset v_r = V$ in an $n$- dimensional vector space $V$. Define a Schubert cell of $G_{r_{k,n}}$ as follows
\begin{equation*}
%\begin{align*}
%\begin{split}
C(\alpha) = \left\{
\begin{split}
W\in G_{r_{k,n}}| \dim W\cap V_j = i \mbox{  if  } a_i\leq j < \alpha_{i+1}\\
\mbox{  where  } 1 \leq j \leq n, ~~ 0\leq i\leq k \mbox{  and  } \alpha_0 = 0
\end{split}
\right\}
\end{equation*}

The closure of the Schubert cell $C(\alpha)$ in $G_{r_{k,n}}$ is called the Schubert  variety corresponding to the index $\alpha$ (or simply the $\alpha^{\mbox{th}}$ schubert variety) denote by $\chi(\alpha)$. As a subscheme of $G_{r_{k,n}}, \chi(\alpha)$ is endowed with its canonical reduced subscheme structure with
$$
\dim\chi(\alpha) = \sum_{i = 1}^r \alpha_i-\frac{r(r+1)}{\alpha}
$$ 
\end{definition}

\begin{proposition}
The homogeneous ideal of a Schubert variety is a Prime ideal.
\end{proposition}
\begin{proof}
For $\alpha$ in $l(k,n)$, since $C(\alpha)$ is an irreducible variety, its closure $\chi(\alpha)$ is irreducible. This together with the fact that $\chi(\alpha)$ is reduced implies that $\chi(\alpha)$ is an integral scheme, which in turn implies that the homogeneous ideal of $\chi(\alpha)$ is a Prime ideal in $\mathbb{K}[x_{\beta}, \beta\in I(k,n)]$ the homogeneous coordinate ring of $\mathbb{P}(\Lambda^rV)$
\end{proof}


\begin{remark}
\normalfont We have $\alpha^{\min} = (1,2,\ldots,r)$ and $\alpha^{\max} = (n-r+1, n-r+2,\ldots, n).$ The $C(\alpha^{\min})$ is a point.
\end{remark}

$$C(\alpha^{\max}) = \cup_{\alpha\max}$$
and it is the only Schubert cell that is open subset of $G_{r_{k,n}}$ which is called the big cell.\\
Since Grassmannian is irreducible, $\chi(\alpha^{\max}) = G_{r_{k,n}}$ i.e, $G_{r_{k,n}}$ itself is a Schubert variety.

\section*{Positroid cell decomposition}
Positroid cells denote as $P(\alpha)$ can be represented by $2\times n$ matrices $A =[v_1,\ldots,v_r], v_i\in\mathbb{R}^2$ with some possible empty subset of zero columns $v_i = 0$ and some (cyclically). Consecutive columns $v_r,v_{r+1},\ldots,v_n$ parallel to each other. Suppose $k = 2$, we have the matrix form
\begin{equation}
\begin{pmatrix}
r & q & p & t & 0 & 0 & z & x\\
0 & 0 & 0 & s & 0 & u & y & 
\end{pmatrix}
 \end{equation}
after the row reduction and deletion of the pivot columns.\\
We have 

\begin{equation}
\begin{pmatrix}
* & * & * & * & 0 & 0 & * & *\\
0 & 0 & 0 & * & 0 & * & * & 
\end{pmatrix}
 \end{equation}
Thus, a blocked zero is the one with dot over it and everything to the left of a blocked zero is zero. It is not allowed to have a 0 with a dot above and a dot to the left.\\
One may assume $A$ has  no zero columns then the combinatorial structure is given by a decomposition of the set $[n]$ into a disjoint union of cyclically consecutive intervals $[n] = B_1\cup\ldots\cup B_r$.\\
Then the Pl$\ddot{\mbox{u}}$cker coordinates $\Delta_{ij}$ is strictly  positive if $i$ and $j$ belong to two different intervals. $B_i$'s and $\Delta_{ij} = 0$ if $i$ and $j$ are in the same interval.

\begin{remark}
 \normalfont The closure of Positroid cells $P(\alpha)$ is called Positroid variety denote by $Q(x)$.
\end{remark}
The classical example of Positroid variety is the Schubert variety $C(\alpha)$ since Grassmannian is a disjoint union of the schubert cells $C(\alpha), ~~\alpha\in I(k,n)$. It will be ideal to say that Grassmannian is a disjoint union of Positroid cells $P(\alpha)$.
The positroid varieties are subvarieties of $G_{r_{k,n}}$ indexed by various posets.

\section*{Combinatorial Description of Positroid ideals in a totally nonnegative $G_{r_{k,n}}$} 

\begin{definition}
\normalfont There are several posets that can index the Positroid cells namely;\\
The Grassmann necklace\\
The Decorative permutation\\
The plabic network\\
The Matroids as well as the Schubert varieties.
\end{definition}

\begin{definition}
\normalfont Let $M$ be a matroid of rank $k$ on $[n]$. Define a sequence of $k$- element subset $J(M) = (J_1,J_2,\ldots,J_n)$ by letting $J_r$ be the minimal base of the matroid and $J = (J_1, J_2, \ldots, J_r)\in Jugg(k,n).$
\end{definition}
\begin{lemma}[3.20 (Pos, Oh)] Let $J\in Jugg(k,n)$. The collection $M_j$ is a matroid.\\
The matroid $M_j:=\left\{I\in \binom{[n]}{k}|J\geq I_r\right\}$ are called positroids.
\end{lemma}
Every positroids is a special matroid that can be represented by totally positive matrices.
\begin{definition}
\normalfont Given a Grassmann necklace $I = (I_1,\ldots,I_n)$ define the positroid
\begin{equation*}
M_I: = \left\{
\begin{split}
J\in \binom{[n]}{k}|I_i\leq J\\
\mbox{ for all } i\in [n]
\end{split}
\right\}
\end{equation*}
\end{definition}
\begin{definition}
\normalfont A Grassmann necklace is a sequence $I = (I_1,\ldots,I_n,I_{n+1} = I_i)$ of $k$- element subset, of $[n]$ such that for all $i\in[n]$

\begin{equation*}
I_{i+1} = \left\{
\begin{split}
I_i|\{i\}\cup\{j\} \mbox{  for some } j \in [n] \mbox{ if } i\in I_1\\\\
I_i~~~~~~~~~~~~~~~~~~~~~~~\mbox{  if  } i\notin I_i
\end{split}
\right\}
\end{equation*}
$I$ is connected if $I_i\neq I_j$ for $i\neq j$
\end{definition}
\begin{theorem}
The homogeneous ideal $J$ of $G_{r_{k,n}}$ is generated by the homogeneous polynomials of the form
\begin{equation}
\sum_{\sigma\in s(r-k,l)}\mbox{sgn} (\sigma)x(\alpha_1,\ldots,\alpha_k,\alpha_{k+1}^{\sigma},\ldots,\alpha_r^\sigma)x(\beta_1^\sigma,\ldots,\beta_l^\sigma,\beta_{1+1},\ldots,\beta_r).~~~~~~1<k<l<r 
\end{equation}
\mbox{  and  }where $k$ and $l$ are fixed integers with for every $\alpha,\beta$ in $I(r,n)$, the above sum runs over all the shufflings of $\{\alpha_{k+1},\alpha_{k+2},\ldots\alpha_r\}$ and $\{\beta_{1},\beta_{2},\ldots,\beta_{l}\}$
\end{theorem}
\begin{proposition}
The Positroid ideal $J$ of $G_{r_{k,n}}\geq0$ is generated by the homogeneous polynomial of the form (3) as in the above theorem 
\end{proposition}
\begin{proof}
Consider the homogeneous coordinate rings $S = k[x_{\alpha},\alpha\in\binom{[n]}{k}]$ and $R = k[p_{\alpha},\alpha\in\binom{[n]}{k}]$ of $\mathbb{P}(\Lambda^rV)$ and $G_{r_{k,n}}\geq0$ respectively.\\
There exist a natural homomorphism
\begin{align*}
\phi:& S\longrightarrow R\\
& x_\alpha \longrightarrow P_{\alpha}
\end{align*}
whose kernel is the ideal $J$. If $J^\prime$ is the ideal generated by the polynomials of the form (3), then it implies that $J^\prime\subset\ker\phi = J$. Hence a surjective homomorphism
\begin{align*}
\Phi:& S/J^{\prime} \longrightarrow R\\
&\bar{x}_{\alpha}\longrightarrow P_\alpha
\end{align*}
Thus we show that $J^\prime = J$ by prooving that $\Phi$ is injective. Let $F$ e any nonzero element of $S/J^{\prime}$ and since $S/J^{\prime}$ is generated by standard monomials, it then seen that $F$ can be written as a Linear combination of distinct standard monomials. Then $\Phi(F)$ is a linear combination of distinct  standard monomials on $G_{r_{k,n}}\geq 0$, sincve standard monomials on  $G_{r_{k,n}}\geq 0$ are linearly independent, it follows that $\Phi(F)\neq 0$ and hence $J = J^\prime$
\end{proof}

\begin{remark}
\normalfont A standard monomial on $G_{r_{k,n}}\geq 0$ of length $m$ is a formal expression of the form
$$
P_{\alpha}(1)P_{\alpha}(2)\ldots P_{\alpha}(m)
$$
where $P_{\alpha}(i)$ are Pl$\ddot{\mbox{u}}$cker coordinates and $\alpha(1),\alpha(2),\ldots,\alpha(m)$ is a standard tableau. Thus two standard monomials are distinct if the corresponding standard tableaus are distinct.
\end{remark}
\begin{proof}
Let $J^\prime$ be the ideal generated by the set $\{P_\alpha| \alpha\geq1\}$ and let $R_\beta$ be the homogeneous coordinate ring of $Q(\beta)$. Then having a natural homomorphism
\begin{align*}
\bar{\Phi}: & R/J^\prime \longrightarrow R_\beta\\
& P_\alpha\longrightarrow P_\alpha|Q(\beta)
\end{align*}
which is surjective.
Since $R$ is generated by standard monomials, so is $R/J^\prime$. Then it follows exactly in the same manner as in above theorem that $\bar{\Phi}$ is injective and hence $J^\prime = J_\beta$.
\end{proof}
\begin{remark}
\normalfont For $n\geq k\geq 0,$ the Grassmannian $G_{r_{k,n}}$ over $R$ is the space of $k$ dimensional linear subspaces of $\mathbb{R}^n$ which can be identified with the space of $k\times n$ matrices form projective coordinates on the Grassmannian called the Pl$\ddot{\mbox{u}}$cker coordinates, that are denoted by $\Delta_{I}$ where $I\in\binom{[n]}{k}$.
The totally nonnegative Grassmannian $G_{r_{k,n}}\geq 0$ which is part of $G_{r_{k,n}}$ is identified with the $k\times n$ matrices whose Pl$\ddot{\mbox{u}}$cker coordinates are all totally non-negative.\\
The dimension of $G_{r_{k,n}}$ is $k(n-k).$
\end{remark}
\section*{Object - to - Object mappings in a $G_{r_{k,n}}\geq 0$}

The following objects are important in the study of Positroid ideals in $G_{r_{k,n}}\geq 0$.
\begin{itemize}
\item The Grassmann necklace
\item The Decorative permutation
\item The Plabic graph/reduced  Plabic graph
\end{itemize}
These objects which  are in one-to-one correspondence with the positroid ideals help to establish a condition of weak seperation of the positroid ideals.

\begin{definition}
\normalfont A decorated permutation $\pi = (\pi, \mbox{col})$ is a permutation $\pi \in \sigma_n$ together with coloring fuction col from the set of fixed points $\{i|\pi(i) = i\}$ to $\{1,-1\}$.\\
For $i,j\in[n], \{i,j\}$ forms an alignment in $\pi$ if $i, \pi(i), \pi(j), j$  are cyclically ordered (and all distict). The number of alignment in $\pi$ is denoted by $al(\pi)$ and the length $l(\pi)$ is defined to be the $k(n-k) + al(\pi).$
\end{definition}
\section{*Linking between a Grassmann necklace and a decorative permutation}
Given a Grassmann necklace $I$, denote $\pi_I = (\pi_1, \mbox{col} I)$ as follows;
\begin{itemize}
\item if $I_{i+1} = I_i|\{i\}\cup\{j\}$ for $i\neq j,$ then $\pi(i) = j$\\
if $I_{i+1} = I_i$ and $i+I_i$  (resp; $i\in I_i$) then $\pi(i) = i$ and $\mbox{col}(i) = 1$ (resp., $\mbox{col}(i) = -1$)
\end{itemize}

\begin{definition}
\normalfont A plabic graph (Planar bicoloured graph) is a planar undirected graph $G$ drawn inside a disk with vertices coloured in black or white colour. The vertices on the boundary vertices are labelled in clockwise order by the elements of $[n]$.  
\end{definition}

\begin{definition}
\normalfont A strand in a plabic graph $G$ is a directed path that satisfies the `` rules of the road'' at every black vertex it makes a sharp right turn, and at every white vertex it makes a sharp left turn.
\end{definition}

\begin{definition}
\normalfont A plabic graph $G$ is called reduced if the following holds.
\begin{itemize}
\item A strand cannot be a closed loop in the interior of $G$.
\item If a strand passes through the same edge twice, then it must be a simple loop that starts and ends at the boundary leaf.
\item Guven any two strands, if they have two edges $e$ and $e^\prime$ in common, then one strand should be directed from $e$ to $e^\prime$ while the other strand should be directed from $e^\prime$ to $e$.\\

Any strand connects two  boundary vertices in a reduced plabic graph $G$, Linking between a decorative permutation and plabic graph.
The associated decorative permutation in a plabic is called a decorated strand permutation denote by $\pi_G = (\pi_G, \mbox{col}_G)$ with $G$ for which $\pi_G(s) = i$ if the strand that starts at a boundary vertex $j$ ends at a boundary vertex $i$,we labeled such strand $i$.
\item if $\pi_G(i) = i$, then $i$ must be connected to a  boundary leaf $v$ and $\mbox{col}(i) = +1$ if $v$ is white and $\mbox{col}(i) = -1$ if $v$ is black. 
\end{itemize}
\end{definition}



\section*{Connected Components of the Objects}
The connected components of these objects namely: the Decorative permutation $\pi$, the Grassmann necklace $I$ and the Positroid $M_I$ will have the subsets of $[n]$ that inherit their cyclic order from $[n]$ as their ground set.

\begin{definition}
\normalfont Let $\pi$ be a decorated permutation\\
Let $[n] = \cup S_i$ be the finest non-crossing partition of $[n]$ such that if $i\in S_j$ then $\pi(i)\in S_j$.\\
Let $\pi(j)$ be the restriction of $\pi$ to the set $S_j$ and let $I(j)$ be the associated Grassmann necklace on the ground set $S_j,$ for $j = 1$.We call $\pi(j)$ the connected components of $\pi$ and $I(j)$ the connected components of $I$. We say that $\pi$ and $I$ are connected if they have exactly one connected component.\\
\textbf{Note:} Each fixed point of $\pi$( of either color) form a connected components.
\end{definition}

\begin{definition}
\normalfont Let $[n] = S_1\cup S_2\cup \cdots \cup S_r$ be a partition of $[n]$ into disjoint subsets. We say that $[n]$ is non-crossung if for any circularly ordered $(a,b,c,d)$ we have $\{a,c\}\subseteq S_i$ and $\{b,d\}\subseteq S_j$ then $i = j$
\end{definition}

\begin{lemma}
The decorative permutation is disconnected if and only if there are two circular intervals $[i,j)$ and $[j,i)$ such that $\pi$ takes $[i,j)$ and $[j,i)$ to themselves.
\end{lemma}
\begin{proof}
If such intervals exist, then the pair $[n] = [i,j)$ and $[j,i)$ is a non-crossing partition preserved by $\pi$. So there is a non trivial non crossing partition preserved by $\pi$ and $\pi$ is not connected.\\
Conversely, any non trivial non crossing permutation can be coarsened to a pair of intervals of this form so if $\pi$ is disconnected, then there is a pair of interval of this form
\end{proof}

\begin{lemma}
A Grassmann necklace $I = (I_1,\ldots,I_n)$ is connected if and only if the sets $I_1,\ldots,I_n$ are all distinct.
\end{lemma}
\begin{proof}
If $\pi$ is disconnected, then let $[i,j)$ and $[j,i)$ be as in above lemma. As we change from $I_i$ to $I_{i+1}$ to $I_{i+2}$ to $\cdots$ to $I_j$, each element of $[i,j)$ is removed once and is added back in once. So $I_i = I_j$.\\
Conversely, suppose that $I_i = I_j$. As we change from $I_i$ to $I_{i+1}$ to $I_{i+2}$ and so forth up to $I_j,$ each element of $[i,j)$ is removed Once. In other to have $I_i = I_j$, each elements of $[i,j)$ must be added back in once. So $\pi$ takes $[i,j)$ to itself.
\end{proof}
\begin{proposition}
\normalfont The following conditions are equivalent
\begin{itemize}
	\item[(i)] Every sequence $I$ of $k$- element subsets of $[n]$ are finitely generated.
	\item[(ii)] Every non-empty set of $I$ in $\binom{[n]}{k}$ has a maximal element.
	\item[(iii)] Every ascending chain (by inclusion) of the set $I = (I_1\subseteq I_2\subseteq \cdots \subseteq I_n)$ is stationary.
\end{itemize}
\end{proposition}

\begin{proof}
\begin{description}
\item[$(i)\Rightarrow (ii)$] Let $\sum_i I$ be the family of every set of sequence $I = (I_1\subseteq I_2\subseteq \cdots \subseteq I_n)$. Since $\sum_i I$ is non-empty, it has a maximal element say $n$.\\
If $I_i\neq I$, consider $I_{i+1}, i\in I, i\notin I$ which is obtain from $I_i$ by deleting $\{i\}$ and adding another element $\{j\}$. This implies is finitely generated , hence a contradiction. Thus, $I_{i+1} = I$, it implies is finitely generated.
\item[$(ii)\Rightarrow (iii)$] Contrarily, if there is a non-empty set $I$ in $\sum_{i}I$ with no maximal element, then inductively we construct a non-terminating sequence in $\sum_i I$ and thus the set $I$ in $\binom{[n]}{k}$ has a maximal element say $I_n$.
\item[$(iii)\Rightarrow (i)$] Let $I_1\subset I_2\subset\cdots$ be an increasing sequence of every $k-$ element subset in $[n]$, then $I =\bigcup_i^n{I_i},$ hence $I = (I_i, I_{i+1},\ldots,I_n)$. This implies that is finitely generated.\\
If $i\in I$, it shows that $I_{i+1} = I_i$ since $I_{i+1}$ is contained in $I_i$ by deleting $\{i\}$ and adding another element $\{j\}$. Continuing in the same manner, we have that $I_{i+1} = I_i \cdots = I_n$ where $n$ is the maximal element in $\sum_{i}(I)$. Hence $I_i = I_n,$ terminates. 
\end{description}
\end{proof}
\begin{proposition}
If a Grassmann necklace $I$ satisfies the above conditions, then the Positroid ideal indexed by $I$ is a Noetherian.
\end{proposition}

\begin{proof}
Suppose Grassmann necklace $I$ satisfies the condition above, then we define the set of Positroid ideals indexed by the Grassmann necklace as follows;
$$
P(I) = \left\{J\in\binom{[n]}{k}| I\leq J\right\}
$$
Since this set is non-empty, it contains the maximal element say $n$, then for every $i\in I$, there exists $j\in j$ such that if $I_i\neq I$ we have $I_{i+1}$ gotten from $I_i$ by deleting $\{i\}$ once and adding $\{j\}$ at most once if $I_{i+1}\neq I$, continue in that manner until we obtain $I_n = I.$ This implies finitely generated. Thus is Noetheriean\\
Conversely, if $P(I)$ is Noetherian then the set of sequence $I = (I_1\subseteq I_2\subseteq\cdots\subseteq I_r)$ with its corresponding positroid $J\in\binom{[n]}{k}$ contains the maximal element such that $I_i\leq J$. If $I_i\neq J$ we obtain $I_{i+1}$ from $I_i$ by deleting an element and adding another element at most once. This implies is finitely generated. Continuity in this manner we have the set of chains $I = I_{i}\longrightarrow I_{i+1}\longrightarrow \cdots \longrightarrow I_n\longrightarrow I_{n+1}\longrightarrow \cdots J.$ Thus $I_n = J$, hence terminates. 
\end{proof}
\section*{Weak separation of the combinatorial objects of study: Grassmann necklace, Decorative permutation, Plabic graph and ideals of Positroid $J$.}
In this section, we try to equip the condition of weak separation of those objects which have s bijective correspondence with the Positroid ideals of $G_{r_{k,n}}\geq 0$. \\
We recall a definition of Grassmann necklace $I = (I_1,\ldots,I_n)$, we define the Positroid ideal $P(I)$ as follows
\begin{equation*}
P(I) = \left[
\begin{split}
J \in \binom{[n]}{k}|I_i\leq J\\
\mbox{  for all  } i \in [n]
\end{split}
\right]
\end{equation*}
for a Grassmann necklace $I = (I_1,\ldots,I_n)$ corresponding with the $P(I)$, a collection $C$ inside the $P(I)$ is said to be weakly separated and $I\subseteq C\subseteq P(I)$, if $C$ is maximal among the weakly separated collections in $P_{I}$, then $C$ is called a maximal weakly separated collection.

\begin{proposition}
For any Grassmann necklace $I$, we have $I\subseteq P(I)$ and $I$ is weakly separated.
\end{proposition}

\begin{proof}
for every $i$ and $j$ in $[n]$, we must show that $I_i\leq I_j$ and $I_i || I_j$. By definition, $I_{k+1}$ is either obtained from $I_k$ by deleting $k$ and adding another element or else $I_{k+1} = I_k$.
As we do the changes
$$
I_1\longrightarrow I_2\longrightarrow\cdots\longrightarrow\cdots\longrightarrow I_n\longrightarrow I_1
$$
we delete each $k\in[n]$ at most once in the transformation $I_k\longrightarrow I_{k+1}.$ This implies that we add each $k$ at most once. Let us show that $I_j|I_i\subseteq [j,i)$. Suppose that this is not true and there exists $k \in (I_j|I_i)\cap[i,j)$.\\
Suppose that this is not true and there exists $k\in (I_j|I_i)\subseteq [j,i)$. Note that $I_{k+1}\neq I_k$ otherwise $k$ belongs to all elements of the Grassmann necklace or $k$ does not belong to all elements of the necklace. Consider the sequence of changes
$$
I_i\longrightarrow I_{i+2}\longrightarrow\cdots\longrightarrow I_k\longrightarrow I_{r+1}\longrightarrow \cdots\longrightarrow I_j
$$ 
we should have $k\notin I_i, k\in I_k, k\notin I_{k+1}, k\in I_j.$ Thus $k$ should be added twice as we go from $I_i$ to $I_k$ and as we go from $I_{k+1}$  to $I_j$. We get a contradiction.Thus, $I_j|I_i\subseteq [J,i)$ and similarly $I_i|I_j\subseteq [i,j]$, we conclude that $I_i\leq i I_j$ and $I_i||I_j$ as desired.
\end{proof}
\begin{definition}
\normalfont Let $\pi$ be a decorated permutation. Let $[n] = \cup S_i$ be the finest non crossing partition of $[n]$ such that if $i\in S_j$ then $\pi(i)\in S_j$.

Let $\pi(J)$ be the restricting of $\pi$ to the set $S_j$ and let $I(j)$ be the associated Grassmann necklace on the ground set $S_j,$ for $j = 1,\ldots,r$. We call $\pi(j)$ the connected components of $\pi$ and $I(j)$ the connected components of $I$. Then we say that $\pi$ and $I$ are connected if they have exactly one connected component.
\end{definition}


\begin{remark}
\normalfont Each fixed point of $\pi$ (of either color) form a connected components. Suppose $I_i = I_j$ for some $i\neq j,$ then we let $I^\prime = [i,j)\cap I_i~~~ J^\prime = j \cap [i,j)~~~~~~~ I^2 = [j,i)\cap I_i~~~ J^2 = J\cap [j,i)~~$ for all $j$ in $M.$\\
$|J| = k$ and $|J\cap [i,j)| = k^i$\\
$|J\cap [j,i)| = k^2$
\end{remark}
 \begin{proposition}
The matroid $M$ is a direct sum of two matroid $M^\prime$  and $M^2$ supported on the ground set $[i,j)$ and $[j,i)$ having rank $k^\prime$ and $k^2$. In otherwords, there are matroids $M^\prime$ and $M^2$ such that $J$ is in $M$ iff $J\cap [i,j)$ is in $M^\prime$ and $J\cap [j,i)$ in $M^2$
\end{proposition}

\begin{proposition}
for $k\in [i,j]$ the set $i_k$ is of the form $J\cup I^2$ for some $J\in M$ for $k\in[J,i)$, the set $I_k$ is of the form $I^\prime\cup J$ for some $J\in M^2.$
\end{proposition}
\begin{proof}
Consider the case that $k\in[i,j]$, the other case is similar. Recall that $I_k$ is the minimal element of $M$ since $M = M^\prime\oplus M^2$. We know that $I_k = J^\prime\cup J^2$ where $J^r$ is the $\leq_k$ minimal element of $M^\prime.$ BUt $[j,i)$ the other $\leq_i$ and $\leq_k$ coincide so $J^2$ is the minimal element of $M^2$ namely $I^2.$
\end{proof}

\begin{lemma}
Let $J^\prime = j^\prime\cup j^2\in M$.\\
If $I$ is weakly separated from $I^\prime\cup I^2,$ then either $I^\prime = J^\prime$ or $I^2 = J^2$.
\end{lemma}

\begin{proof}
Suppose on the contrary, that $J^\prime\neq I^\prime$ and $J^2\neq I^2$. Since $I^\prime \leq_i J^\prime$ there are $a$ and $b\in [i,j)$ with $i\leq_i a\leq_i b$ such that $a\in I^\prime|J^\prime$ and $b\in J^\prime|I^\prime$.\\
Similarly there are $c$ and $d\in [j,i)$ with $i\leq_j c\leq_j d$ such that $c\in I^2| J^2$ and $d$ in $J^2|I^2$.\\
Then $a$ and $c$ are in $I^\prime\cup I^2|J^\prime\cup J^2$ while $b$ and $d$ are in $J^\prime\cup J^2|I^\prime\cup I^2.$ So $I^\prime\cup I^2$ and $j$ are not weakly separated.
\end{proof}

\begin{proposition}
If $C$ is a weakly separated collection in $M$, then there are weakly separated collections $c^\prime$ and $c^2$ in $M^\prime$ and $M^2$\\
such that\\
$$
C = \{J\cup I^2; J\in C^\prime\cup\{I^\prime\cup J; J\in C^2\}\}
$$
Conversely, if $c^\prime$ and $c^2$ are weakly separated collections in $	M^\prime$ and $M^2$, then the above formula defines a weakly separated collection in  $M$. The collection $C$ is maximal if and only if $c^\prime$ and $c^2$ are.
\end{proposition}

\begin{proof}
First, suppose that $C$ is a weakly separated collection in $M$. Since $I\in C$, we have that $I^\prime\cup I^2\in C.$ Since every $j\in C$ is either of the form $J^\prime\cup I^2$ or $I^\prime\cup J^2$. Let $C^r$ be the collection of all sets $J^r$ for which $J^r\cup I^{s-r}$ is in $C$. The condition that $C$ is weakly separated implies that $C^r$ is the condition that $I\subseteq C\subseteq M$  implies $I^r\subseteq C^r\subseteq M^r$. So $C^r$ is a weakly separated collection in $M^r$ and it is clear that $C$ is brought from $C^\prime$ and $C^2$ in the indicated manner.\\
Conversely, it is easy to check that if $c^\prime$ and $c^2$ are weakly separated collections in $M^\prime$ and $M^2,$ then the above formula gives a weakly separated collection in $M$.\\
Finally, if $C\subseteq C^\prime$ with $C^\prime$ a weakly separated collection in $M$, then either $c^\prime\subseteq (c^\prime)$ or $c^2\subseteq(c^\prime)$. So if $c$ is not maximal either $c^\prime$ or $c^2$ is not.
The converse is similar.
\end{proof}
\end{document}
